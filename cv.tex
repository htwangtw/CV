%%%%%%%%%%%%%%%%%%%%%%%%%%%%%%%%%%%%%%%%%
% Medium Length Professional CV
% LaTeX Template
% Version 2.0 (8/5/13)
%
% This template has been downloaded from:
% http://www.LaTeXTemplates.com
%
% Original author:
% Trey Hunner (http://www.treyhunner.com/)
%
% Modified by Hao-Ting Wang
%
% Important note:
% This template requires the resume.cls file to be in the same directory as the
% .tex file. The resume.cls file provides the resume style used for structuring the
% document.
%
%%%%%%%%%%%%%%%%%%%%%%%%%%%%%%%%%%%%%%%%%

%----------------------------------------------------------------------------------------
%	PACKAGES AND OTHER DOCUMENT CONFIGURATIONS
%----------------------------------------------------------------------------------------

\documentclass{resume} % Use the custom resume.cls style
\usepackage{tabularx}
\usepackage{doi}
\usepackage{hyperref}
\hypersetup{
  colorlinks=true,
  linkcolor=blue,
  filecolor=magenta,
  urlcolor=cyan,
}

\usepackage[resetlabels]{multibib}
% Define bibliographies.
\newcites{p,j,h,c,g}{
 \normalsize{Preprint},
 \normalsize{Peer-Reviewed Journals},
 \normalsize{Highlights},
 \normalsize{Conference Posters},
 \normalsize{Consortium}}

\usepackage{textcomp}
\usepackage[left=0.4in,top=0.4in,right=0.4in,bottom=0.75in]{geometry} % Document margins
\usepackage{enumitem}

\name{Hao-Ting Wang, PhD} % Your name
\position{Postdoctoral Researcher in cognitive computational neuroscience}
\address{Centre de recherche de l'Institut universitaire de g\'eriatrie de Montr\'eal (CRIUGM)}
\address{Montr\'eal, Queb\'ec, Canada}
%\address{htwangtw@gmail.com \\ https://github.com/htwangtw}

\begin{document}

%----------------------------------------------------------------------------------------
%	Research overview
%----------------------------------------------------------------------------------------
%\begin{rSection}{Research interest}
%My research focuses on multivariate analysis methods combined with machine learning techniques to explore human fMRI and cognitive performance in dataset with $N > 100$.
%
%\end{rSection}

%
%----------------------------------------------------------------------------------------
%	RESEARCH
%----------------------------------------------------------------------------------------

\begin{rSection}{Research positions}

  \begin{rSubsection}{Postdoctral Researcher}{Sept. 2021 -- Present}{Centre de recherche de l'Institut universitaire de g\'eriatrie de Montr\'eal (CRIUGM)}{Montr\'eal, QC, Canada}
    \item Principal Investigator: Prof Pierre Bellec
    \item Software for neuroimaging and neurodegeneration biomarker discovery.
  \end{rSubsection}

  \begin{rSubsection}{Research Fellow}{Sept. 2019 -- Aug. 2021}{Sackler Centre for Consciousness Science, University of Sussex}{Brighton, United Kingdom}
    \item Principal Investigators: Prof Hugo Critchley, Prof Sarah Garfinkle
    \item Cognitive processes in psychiatric conditions with neuroimaging and physiology measures.
  \end{rSubsection}

  \begin{rSubsection}{Postdoctoral Research Associate}{Nov. 2018 -- Aug. 2019}{University of York}{York, United Kingdom}
  \item Principal Investigator: Prof Jonathan Smallwood
  \item Working on the European Research Council funded project---Wandering Minds
  \end{rSubsection}

  \begin{rSubsection}{Research Administrator}{Oct. 2015 -- Oct. 2018}{University of York}{York, United Kingdom}
    \item Principal Investigators: Prof Jonathan Smallwood and Prof Elizabeth Jefferies
    \item Experiment design, project management, neuroimaging analysis pipeline development
  \end{rSubsection}

  \end{rSection}

%%----------------------------------------------------------------------------------------
%	EDUCATION SECTION
%----------------------------------------------------------------------------------------

\begin{rSection}{Education}

  \begin{rSubsection}{PhD in Cognitive Neuroscience and Neuroimaging}{Sept. 2015 -- Dec. 2018}{University of York}{York, United Kingdom}
    \item Supervisors: Prof Jonathan Smallwood and Prof Elizabeth Jefferies
    \item Thesis: ``\textit{Towards an Ontology of Ongoing Thought}''
  \end{rSubsection}

  \begin{EDUrSubsection}{Master of Research in Psychology}{Sept. 2013 -- Sept. 2014}{University of York}{York, United Kingdom}
  \end{EDUrSubsection}

  \begin{EDUrSubsection}{BSc in Psychology}{Sept. 2009 -- June 2013}{National Chengchi University}{Taipei, Taiwan}
  \end{EDUrSubsection}
\end{rSection}


%----------------------------------------------------------------------------------------
%	GRANTS SECTION
%----------------------------------------------------------------------------------------

\begin{rSection}{Awards and Scholarships}
  
  \begin{rSubsection}{Scholarships}{}{}{}
    \item 2022 - 2024 Institut de valorisation des données (IVADO): Postdoctoral scholarship. QC, Canada. \textbf{(CAD\$70,000)}
    \item 2022 UNIQUE: UNIQUE Excellence Scholarship. QC, Canada. (CAD\$20,000; declined)
    \item 2019 - 2021 Sackler Foundation: postdoctral research fellowship. Brighton, United Kingdom \textbf{(\pounds 33,199 per annum)}
  \end{rSubsection}
  
  \begin{rSubsection}{Awards}{}{}{}
  	\item 2023 Neuro-Irv and Helga Cooper Foundation Open Science Prizes: Canadian Trainee Prize. \textbf{(CAD\$5,000)}
    \item 2017 Guarantors of Brain Travel Award: Machine Learning Summer School. T\"{u}bingen, Germany (\pounds 600)
    \item 2016 The Neuro Bureau Travel Award: Brainhack Vienna. Vienna, Austria (USD\$500)
    \item 2014 University of York Department Summer Bursary Award. York, United Kingdom (\pounds 1000)
  \end{rSubsection}

\end{rSection}
\pagebreak

%----------------------------------------------------------------------------------------
%	PUBLICATIONS
%----------------------------------------------------------------------------------------
\begin{rSection}{Recent Publications}

	% \bibliographystyleh{IEEEtran}
	% \nociteh{*}
	% \bibliographyh{highlights}
  
  \bibliographystylep{IEEEtran}
  \nocitep{*}
  \bibliographyp{preprints}

%  \bibliographystyleg{IEEEtran}
%  \nociteg{*}
%  \bibliographyg{consortium}
  
  \bibliographystylej{IEEEtran}
  \nocitej{*}
  \bibliographyj{publications}

  \bibliographystylec{IEEEtran}
  \nocitec{*}
  \bibliographyc{conferences}

\end{rSection}

%%----------------------------------------------------------------------------------------
%	Talks
%----------------------------------------------------------------------------------------

\begin{rSection}{Talks}
  \begin{tabular}{@{} l l @{\hspace{10ex}}}
    2024 & Symposim Speaker. Big Data in Psychiatry: Does It Solve the Reproducibility Problem? \\ & 2024 Society of Biological Psychiatry. Austin, TX, USA.\\
    2023 & Canadian Trainee Prize recipient talk. 5th Annual Neuro Open Science in Action Symposium.  \\& Montreal, QC, Canada.\\
    2023 & Hackathon and Early Career Development. OHBM 2023. Montreal, QC, Canada.\\
    2023 & Fantastic open source projects and how to find them. UNIQUE Student Symposium 2023. Montreal, QC, Canada.\\
    2023 & A reproducible benchmark of fMRI denoising strategies in fMRIPrep and Nilearn. \\ & Q-BIN Science Day. Quebec City, QC, Canada. 2023\\
   	2022 & load\_confounds. Neuroimaging in Montreal. Montreal, QC, Canada.\\
    2021 & Panel speaker on neuroinformatics at University of Texas Brainstorms\\
    2021 & Panel speaker on academic career at MAIN 2021\\
    2021 & Panel speaker at SciPy2021 Biology and Neuroscience mini-symposium\\
  	2021 & Canonical correlation analysis application in neuroimaging data, Queen's University, Kingston, Canada\\
    2019 & Recent trend in resting-state functional connectivity, University of Sussex, Brighton, UK\\
    2019 & Data simulation workshop, University of York, York, UK\\
    2019 & Multivariate mapping of functional brain and behaviour, Child Mind Institute, New York, USA\\
    2018 & Small steps to reproducible science, University of York, York, UK\\
  \end{tabular}
\end{rSection}

%----------------------------------------------------------------------------------------
%	Academic Service
%----------------------------------------------------------------------------------------
\begin{rSection}{Professional service}

\textbf{Committees}

\begin{tabular}{@{} l l @{\hspace{6ex}}}
  Oct. 2022 -- June 2023 & Brainhack school organiser, Montreal, QC, Canada \\
  Oct. 2021 -- Sep. 2022 & Hackathon Chair, Open Science special interest group, OHBM, Glasgow, UK \\
  Mar. 2020 -- Aug. 2021 & ECR representative, Sussex Neuroscience Steering Committee, University of Sussex\\
  Jun. 2021 & OHBM Sparkle special task force, OHBM, virtual. \\
  Jun. 2021 & Live Q \& A cohost and general enquiry, OHBM Brainhack, virtual. \\
  Jun. 2020 & Teaching assistant, OHBM Brainhack, virtual. \\
  Oct. 2018 -- Aug. 2019 & Member, Open Science Interest Group, University of York\\
  Oct. 2018 -- Aug. 2019 & Member, Early Career Researcher forum, University of York\\
  Mar. 2017 & Organizing committee, Brainhack York, York, UK.\\
\end{tabular}

\textbf{Open source software}

\begin{tabular}{@{} l l l @{\hspace{6ex}}}
  2021 -- present & \href{https://github.com/nilearn/nilearn}{NiLearn} & Core developer.\\
  2021 & \href{https://github.com/SIMEXP/load_confounds}{load\_confounds} & Core developer.\\
  2020 & \href{https://github.com/brainhackorg/brainhack_jupyter_book}{Brainhack book} & Contributor and maintainer. \\
  2020 - 2021 & \href{https://github.com/nipype/pydra-fsl}{Pydra-FSL} & Contributor and maintainer. \\
  2019 & \href{https://github.com/nipy/nibabel/pull/793}{NiBable} & Contributor and reviewer.\\
\end{tabular}

\end{rSection}

\begin{rSection}{Membership}
Organization of Human Brain Mapping (OHBM); Open Science Special Interest Group, OHBM.
\end{rSection}

\begin{rSection}{Ad-hoc Peer Review}
Aperture Neuro,
Advances in Methods and Practices in Psychological Science,
Brain Imaging and Behavior,
Communications Biology,
Journal of Open Source Software,
NeuroImage,
Neuroinformatics,
Neurobiology of Aging
\end{rSection}
\pagebreak

%----------------------------------------------------------------------------------------
%	Professional development
%----------------------------------------------------------------------------------------
\begin{rSection}{Professional development}

\begin{tabular}{@{} l l @{\hspace{6ex}}}
  Aug. 2019 & Neurohackademy, Seattle, USA.\\
  Dec. 2017 & Large-scale trends in cortical organization, Leipzig, Germany.\\
  June 2017 & Machine Learning Summer School, T\"{u}bingen, Germany.\\
  Sep. 2016 & Brainhack Vienna, Vienna, Austria.\\
  Feb. 2016 & Brainhack@Paris, Paris, France.\\
\end{tabular}
\end{rSection}



%----------------------------------------------------------------------------------------
%	SUPERVISION
%----------------------------------------------------------------------------------------

\begin{rSection}{Mentoring experience}
  \begin{tabular}{@{} >{}l >{}l >{}l l @{\hspace{10ex}}}
    \textbf{PhD} &  & \\
    2019--2021 & Will Strawson & University of Sussex (with Prof. Sarah Garfinkle) \\
    \textbf{MSc} &  & \\
    2019 & Bronte McKeown, Will Strawson & University of York (with Prof. Jonathan Smallwood)\\
    2018 & Delali Konu, Rebecca Lowndes & University of York (with Dr. Charlotte Murphy and Prof. Jonathan Smallwood)\\
  \end{tabular}
\end{rSection}


%----------------------------------------------------------------------------------------
%	TEACHING EXPERIENCE SECTION
%----------------------------------------------------------------------------------------

\begin{rSection}{Teaching experience}

\begin{tabular}{@{} l l @{\hspace{6ex}}}
  Dec. 2022 & Instructor, UNIQUE educational workshop, Montreal, Canada.\\
  July 2022 & Teaching assistant, Brainhack School, Montreal, Canada.\\
  Nov. 2021 & Instructor, UNIQUE educational workshop, Montreal, Canada.\\
  June 2020 & Teaching assistant, OHBM Brainhack, Virtual.\\
  Oct. -- Mar. 2016 & Teaching assistant, Programming in Neuroimaging, University of York, York, United Kingdom.\\
\end{tabular}
\end{rSection}


%----------------------------------------------------------------------------------------
%	TECHNICAL STRENGTHS SECTION
%----------------------------------------------------------------------------------------
\begin{rSection}{Technical Expertise}
  \textbf{Overview}: Functional magnetic resonance imaging,
  neuroinformatics, multivariate analysis.

\begin{rSubsection}{Technologies}{}{}{}
  \item \textit{Neuroimaging}: FSL, fMRIPrep, Freesurfer, Connectome Workbench,
                      Brain Image Data Structure (BIDS), nipype
  \item \textit{Statistics}:  nilearn, scikit-learn, JASP
  \item \textit{Deepl learning}:  PyTorch
  \item \textit{Experiment design}: PsychoPy
  \item \textit{Research computing}: container (docker, singularity),
                            cluster computing (SGE, SLRUM),
                            version control (git, github), workflow management (Hydra)
\end{rSubsection}
\begin{rSubsection}{Programming Languages}{}{}{}
  \item Proficient: Python, shell. Competent: \LaTeX, MATLAB. Familiar: R, JavaScript, TypeScript.
\end{rSubsection}

\end{rSection}

%----------------------------------------------------------------------------------------
%	References
%----------------------------------------------------------------------------------------
\begin{rSection}{References}


	Prof. Pierre Bellec\\
	Centre de recherche de l'institut Universitaire de gériatrie de Montréal (CRIUGM), Montreal, Quebec, Canada.\\
 	\href{pierre.bellec@criugm.qc.ca}{pierre.bellec@criugm.qc.ca}
 	
	Prof. Bertrand Thirion\\
	Université Paris Saclay - INRIA, France. \\
 	\href{bertrand.thirion@inria.fr}{bertrand.thirion@inria.fr}
 	
 	Prof. Jonathan Smallwood\\
 	Department of Psychology, Queen's University, Kingston, Ontario, Canada.\\
 	\href{jonathan.smallwood@queensu.ca}{jonathan.smallwood@queensu.ca}

 	Prof. Elizabeth Jefferies\\
 	Department of Psychology, University of York, UK\\
 	\href{beth.jefferies@york.ac.uk}{beth.jefferies@york.ac.uk}

	

\end{rSection}

\sectionskip
\centering
Last updated: \today

\end{document}
