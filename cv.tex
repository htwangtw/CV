%%%%%%%%%%%%%%%%%%%%%%%%%%%%%%%%%%%%%%%%%
% Medium Length Professional CV
% LaTeX Template
% Version 2.0 (8/5/13)
%
% This template has been downloaded from:
% http://www.LaTeXTemplates.com
%
% Original author:
% Trey Hunner (http://www.treyhunner.com/)
%
% Modified by Hao-Ting Wang
%
% Important note:
% This template requires the resume.cls file to be in the same directory as the
% .tex file. The resume.cls file provides the resume style used for structuring the
% document.
%
%%%%%%%%%%%%%%%%%%%%%%%%%%%%%%%%%%%%%%%%%

%----------------------------------------------------------------------------------------
%	PACKAGES AND OTHER DOCUMENT CONFIGURATIONS
%----------------------------------------------------------------------------------------

\documentclass{resume} % Use the custom resume.cls style
\usepackage{tabularx}

\usepackage{hyperref}
\hypersetup{
  colorlinks=true,
  linkcolor=blue,
  filecolor=magenta,
  urlcolor=cyan,
}
\usepackage[resetlabels]{multibib}
% Define bibliographies.
\newcites{j,h,c,g}{
 \normalsize{Peer-Reviewed Journals},
 \normalsize{Highlights},
 \normalsize{Conference Posters},
 \normalsize{Consortion}}

\usepackage{textcomp}
\usepackage[left=0.4in,top=0.4in,right=0.4in,bottom=0.75in]{geometry} % Document margins
\usepackage{enumitem}

\name{Hao-Ting Wang, PhD} % Your name
\position{Postdoctral Researcher in neuroimaging and neuroinformatics}
\address{Centre de recherche de l'Institut universitaire de g\'eriatrie de Montr\'eal (CRIUGM)}
\address{Montr\'eal, Queb\'ec, Canada}
%\address{htwangtw@gmail.com \\ https://github.com/htwangtw}

\begin{document}

%----------------------------------------------------------------------------------------
%	Research overview
%----------------------------------------------------------------------------------------
%\begin{rSection}{Research interest}
%My research focuses on multivariate analysis methods combined with machine learning techniques to explore human fMRI and cognitive performance in dataset with $N > 100$.
%
%\end{rSection}


%----------------------------------------------------------------------------------------
%	RESEARCH
%----------------------------------------------------------------------------------------

\begin{rSection}{Research positions}

  \begin{rSubsection}{Postdoctral Researcher}{Sept. 2021 -- Present}{Centre de recherche de l'Institut universitaire de g\'eriatrie de Montr\'eal (CRIUGM)}{Montr\'eal, QC, Canada}
    \item Principal Investigators: Prof Pierre Bellec, Prof Louis De Beaumont
    \item Data infrastructure for neuroimaging research and Alzheimer's neural biomarker discovery.
  \end{rSubsection}

  \begin{rSubsection}{Research Fellow}{Sept. 2019 -- Aug. 2021}{Sackler Centre for Consciousness Science, University of Sussex}{Brighton, United Kingdom}
    \item Principal Investigators: Prof Hugo Critchley, Prof Sarah Garfinkle
    \item Cognitive processes in psychiatric conditions with neuroimaging and physiology measures.
  \end{rSubsection}

  \begin{rSubsection}{Postdoctoral Research Associate}{Nov. 2018 -- Aug. 2019}{University of York}{York, United Kingdom}
  \item Principal Investigator: Prof Jonathan Smallwood
  \item Working on the European Research Council funded project---Wandering Minds
  \end{rSubsection}

  \begin{rSubsection}{Research Administrator}{Oct. 2015 -- Oct. 2018}{University of York}{York, United Kingdom}
    \item Principal Investigators: Prof Jonathan Smallwood and Prof Elizabeth Jefferies
    \item Experiment design, project management, neuroimaging analysis pipeline development
  \end{rSubsection}

  \end{rSection}

%----------------------------------------------------------------------------------------
%	EDUCATION SECTION
%----------------------------------------------------------------------------------------

\begin{rSection}{Education}

  \begin{rSubsection}{PhD in Cognitive Neuroscience and Neuroimaging}{Sept. 2015 -- Dec. 2018}{University of York}{York, United Kingdom}
    \item Supervisors: Prof Jonathan Smallwood and Prof Elizabeth Jefferies
    \item Thesis: ``\textit{Towards an Ontology of Ongoing Thought}''
  \end{rSubsection}

  \begin{EDUrSubsection}{Master of Research in Psychology}{Sept. 2013 -- Sept. 2014}{University of York}{York, United Kingdom}
  \end{EDUrSubsection}

  \begin{EDUrSubsection}{BSc in Psychology}{Sept. 2009 -- June 2013}{National Chengchi University}{Taipei, Taiwan}
  \end{EDUrSubsection}
\end{rSection}


%----------------------------------------------------------------------------------------
%	GRANTS SECTION
%----------------------------------------------------------------------------------------

\begin{rSection}{Awards}

  \begin{tabular}{@{} c l l r @{\hspace{6ex}}}

  2017 & Travel Award & Guarantors of Brain & \pounds 600\\
  2016 & Travel Award & Brainhack Vienna & \$500\\
  2014 & Department Summer Bursary Award & University of York &\pounds 1000\\

  \end{tabular}

\end{rSection}


%----------------------------------------------------------------------------------------
%	PUBLICATIONS
%----------------------------------------------------------------------------------------
\begin{rSection}{Selected Publications}

	% \bibliographystyleh{IEEEtran}
	% \nociteh{*}
	% \bibliographyh{highlights}

	\bibliographystyleg{IEEEtran}
  \nociteg{*}
  \bibliographyg{consortion}

	\bibliographystylej{IEEEtran}
  \nocitej{*}
  \bibliographyj{publications}

  \bibliographystylec{IEEEtran}
  \nocitec{*}
  \bibliographyc{conferences}

\end{rSection}

%----------------------------------------------------------------------------------------
%	Talks
%----------------------------------------------------------------------------------------

\begin{rSection}{Invited talks}
  \begin{tabular}{@{} l l @{\hspace{10ex}}}
    2021 & Panel speaker at SciPy2021 Biology and Neuroscience mini-symposium\\
  	2021 & Canonical correlation analysis application in neuroimaging data, Queen's University, Kingston, Canada\\
    2019 & Recent trend in resting-state functional connectivity, University of Sussex, Brighton, UK\\
    2019 & Data simulation workshop, University of York, York, UK\\
    2019 & Multivariate mapping of functional brain and behaviour, Child Mind Institute, New York, USA\\
    2018 & Small steps to reproducible science, University of York, York, UK\\
  \end{tabular}
\end{rSection}


%----------------------------------------------------------------------------------------
%	Software
%----------------------------------------------------------------------------------------
\begin{rSection}{Open source software contributions}
  \begin{itemize}
      \item \href{https://github.com/nilearn/nilearn}{NiLearn}: Core developer.
      \item \href{https://github.com/SIMEXP/load_confounds}{load\_confounds}: Core developer.
      \item \href{https://github.com/brainhackorg/brainhack_jupyter_book}{Brainhack book}: csv to markdown table parser for website and code review.
      \item \href{https://github.com/nipype/pydra-fsl}{Pydra-FSL}: FSL wrapped with python workflow engine; nipype 1 to pydra interface converter.
      \item \href{https://github.com/nipy/nibabel/pull/793}{NiBable}: GIFTI data reading method
  \end{itemize}
\end{rSection}


%----------------------------------------------------------------------------------------
%	TECHNICAL STRENGTHS SECTION
%----------------------------------------------------------------------------------------
\begin{rSection}{Technical Expertise}
  \textbf{Overview}: Functional magnetic resonance imaging,
  neuroinformatics, multivariate analysis.

\begin{rSubsection}{Technologies}{}{}{}
  \item \textit{Neuroimaging}: FSL, fMRIPrep, Freesurfer, Connectome Workbench,
                      Brain Image Data Structure (BIDS), nipype
  \item \textit{Statistics}:  nilearn, scikit-learn, JASP
  \item \textit{Experiment design}: PsychoPy
  \item \textit{Research computing}: container (docker, singularity),
                            cluster computing (SGE),
                            version control (git, github)
\end{rSubsection}
\begin{rSubsection}{Programming Languages}{}{}{}
  \item Proficient: Python2/3, shell. Competent: \LaTeX, MATLAB. Familiar: R, JavaScript.
\end{rSubsection}

\end{rSection}


%----------------------------------------------------------------------------------------
%	SUPERVISION
%----------------------------------------------------------------------------------------

\begin{rSection}{Mentoring experience}
  \begin{tabular}{@{} >{}l >{}l >{}l l @{\hspace{10ex}}}
    \textbf{PhD} &  & \\
    2019--2021 & Will Strawson & University of Sussex (with Prof. Sarah Garfinkle) \\
    \textbf{MSc} &  & \\
    2019 & Bronte McKeown, Will Strawson & University of York (with Prof. Jonathan Smallwood)\\
    2018 & Delali Konu, Rebecca Lowndes & University of York (with Dr. Charlotte Murphy and Prof. Jonathan Smallwood)\\
  \end{tabular}
\end{rSection}


%----------------------------------------------------------------------------------------
%	TEACHING EXPERIENCE SECTION
%----------------------------------------------------------------------------------------

\begin{rSection}{Teaching experience}

  \begin{rSubsection}{OHBM Brainhack}{June 2020}{}{}
    \item Brain Image Data Structure teaching assistant.
  \end{rSubsection}

  \begin{rSubsection}{University of York}{October -- March 2016}{Programming in Neuroimaging}{York, United Kingdom}
    \item Teaching assistant: Basic Python, data visualisation, PsychoPy, data analysis, and shell scripting.
  \end{rSubsection}

\end{rSection}

%----------------------------------------------------------------------------------------
%	Professional development
%----------------------------------------------------------------------------------------
\begin{rSection}{Professional development}

\begin{tabular}{@{} l l @{\hspace{6ex}}}
  Aug. 2019 & Neurohackademy, Seattle, USA.\\
  Dec. 2017 & Large-scale trends in cortical organization, Leipzig, Germany.\\
  June 2017 & Machine Learning Summer School, T\"{u}bingen, Germany.\\
  Sep. 2016 & Brainhack Vienna, Vienna, Austria.\\
  Feb. 2016 & Brainhack@Paris, Paris, France.\\
\end{tabular}
\end{rSection}

%----------------------------------------------------------------------------------------
%	Academic Service
%----------------------------------------------------------------------------------------
\begin{rSection}{Professional service}
\begin{tabular}{@{} l l @{\hspace{6ex}}}
 Oct. 2021 -- Present & Hackathon Chair, Open Science special interest group, Organization of Human Brain Mapping \\
  Mar. 2020 -- Aug. 2021 & ECR representative, Sussex Neuroscience Steering Committee, University of Sussex\\
  Jun. 2021 & OHBM Sparkle special task force, OHBM, virtual. \\
  Jun. 2021 & Live Q \& A cohost and general enquiry, OHBM Brainhack, virtual. \\
  Jun. 2020 & Teaching assistant, OHBM Brainhack, virtual. \\
  Oct. 2018 -- Aug. 2019 & Member, Open Science Interest Group, University of York\\
  Oct. 2018 -- Aug. 2019 & Member, Early Career Researcher forum, University of York\\
  Mar. 2017 & Organizing committee, Brainhack York, York, UK.\\
\end{tabular}
\end{rSection}

\begin{rSection}{Ad-hoc Peer Review}
Advances in Methods and Practices in Psychological Science,
Brain Imaging and Behavior,
Journal of Open Science Software,
NeuroImage,
Neuroinformatics,
Neurobiology of Aging
\end{rSection}

\begin{rSection}{Membership}
Organization of Human Brain Mapping
\end{rSection}

%----------------------------------------------------------------------------------------
%	References
%----------------------------------------------------------------------------------------
% \newpage
% \begin{rSection}{References}
% 	Prof. Jonathan Smallwood\\
% 	Department of Psychology, University of York, UK\\
% 	jonny.smallwood@york.ac.uk

% 	Prof. Elizabeth Jefferies\\
% 	Department of Psychology, University of York, UK\\
% 	beth.jefferies@york.ac.uk

% 	Prof. Danilo Bzdok\\
% 	Department of Psychiatry, Psychotherapy and Psychosomatics, RWTH Aachen University, Germany\\
% 	danilo.bzdok@rwth-aachen.de

% \end{rSection}

\sectionskip
\centering
Last updated: \today

\end{document}
